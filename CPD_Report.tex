% Please do not change the document class
\documentclass{scrartcl}

% Please do not change these packages
\usepackage[hidelinks]{hyperref}
\usepackage[none]{hyphenat}
\usepackage{setspace}
\doublespace

% You may add additional packages here
\usepackage{amsmath}

% Please include a clear, concise, and descriptive title
\title{Improving Team-Related Skills}

% Please do not change the subtitle
\subtitle{COMP230 - CPD Report}

% Please put your student number in the author field
\author{1702208}

\begin{document}

\maketitle

\section{Introduction}

During the first year, I have had difficulty keeping myself organized. This includes putting enough time and effort into assignments, communicating with the team and managing my time for all of this. However, over the first semester, I've improved a little bit. Although, this is surely not enough to not feel frustrated about different assignments and other plans. Therefore, I will try to improve at least one of my skills that I'm going to talk about using SMART action.

\section{Interpersonal Domain}

Communicating with people is still my one of the worst skills I have. This is not just being funny in a conversation or not being awkward, but expressing my ideas in general. It's always hard for me to get used to new people, let alone communicate with them about a project. I usually need some time to start talking to others casually, but at the same time, this makes it more and more awkward for me to start casually talking as I feel like I've been left behind and there's barely any chance coming back. This sounds weird, but this is how I feel and that is why I want to get rid of this feeling first and foremost. The first couple of months I had this problem, but near the end of the first semester, I started communicating with the team bit by bit. I still feel a bit left out, but it's not as strong of a feeling as it was before because I know they will accept me. So, it might not be SMART action, but nevertheless, I will keep pushing myself to speak out in meetings and just chatting with the team in general.

\section{Dispositional Domain}

During the first semester, I've been frustrating about assignments due to barely any time-management in terms of consistent workflow. This left me with negative feelings, which I'm going to talk about in the next section. But for this section, the problem is my workflow management. I was having big problems working on team project as I couldn't get myself into working consistently on every assignment. Due to this bad management, I was concentrating more on personal stuff, which is not ideal. Even though personal assignments are also important, a team project is crucial for my portfolio. Therefore, for this self-improvement, I would like to keep a calendar of my working times and stick to it. Write down what time I will be working on assignments and other projects and that will be my free time. And for the first couple of weeks, I will try and keep at least one hour of consistent daily workflow, so that I can get used to it. After getting used to this system I will try increasing working hours bit by bit until I get satisfied in the workflow.

\section{Affective Domain}

Previously mentioned negative feelings such as frustration and stress were caused due to not having done enough work for the team project. I've been struggling having motivation for the game even though I understand the importance to put as much effort to this project as to other assignments. This made me feel like a total disaster in relation to the team, which extremely decreased my self-confidence. Although I perfectly understood that, I still didn't take any serious actions towards improving myself, which made it even worse. I am a very self-critical person, but also very lazy. So even knowing about my low productivity I kept barely doing anything about it while also cursing myself. Strangely enough, I am used to that, which is not a good thing. And mainly because of that I will be doing my best to prevent it from happening in the next semester. So, to decrease my frustration and stress and increase self-confidence, I will try to keep doing at least some work, but for all assignments using the calendar, which I mentioned in the previous section.

\section{Cognitive Domain}

In the first semester, I've mainly been concentrating on one aspect of game development - artificial intelligence. It is something I'm interested in and want to improve my skills at programming AI, but at the same time, I shouldn't focus on just one aspect. Learning multiple elements of programming will definitely benefit as it not only shows one's ability to make different things but also improves already existing ones. I do enjoy finding out new and advanced AI, but I almost never try making something myself, which is a huge mistake. But during our AR/VR prototype making, I really enjoyed the process thus realizing that I should try something different than just artificial intelligence. Therefore, in the next semester, I will try to find something I enjoy making so that I can be more useful to the team while also expanding my knowledge. It will take a lot of effort finding it because it's a process of trial and error. There's hardly anything one can learn about without making mistakes or failing.

\section{Procedural Domain}

Programming skills really depend on a person's debugging ones. It is extremely hard to find a suddenly occurred problem among a huge amount of code. Therefore, it takes a lot of necessary time before applying a solution, which gives frustration. Debugging, nevertheless, helps to find the problem much quicker, which means less time spent on searches. I realized that I lack this skill when problems occurred in one of the assignments. Even though the program didn't work as intended, I continued adding new functions before trying to fix it, which was a huge mistake. When I finally decided to start fixing it, I didn't know where to start. This is where I understood the importance of debugging. Luckily, one of my peers guided me through some very useful features, so that I could try it myself. There were multiple issues there, but eventually, I managed to fix them all. Due to this very mistake of mine, I will be using debugging as soon as I encounter a problem. This way I don't get stuck in one place forever.

\section{Conclusion}

Overall, the challenges I want to overcome are related to the team project. I didn't have many problems related to the individual work, but because of that, all the obstacles I encountered were to do with the team game. Therefore, to prevent somewhat prevent that from happening in the next semester I intend to take actions that I listed above. It will surely take a lot of time and effort to get used to a new system of mine, but it will surely be very valuable.

\bibliographystyle{ieeetran}
\bibliography{references}

\end{document}
